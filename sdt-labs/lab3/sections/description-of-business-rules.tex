\section{Описание бизнес-правил}\label{sec:description-of-business-rules}

\setlength{\parskip}{1.0ex}
\renewcommand{\arraystretch}{2}
\small

\begin{xltabular}[h]{\textwidth}{p{0.1 \textwidth}Xp{0.15 \textwidth}p{0.15 \textwidth}p{0.15 \textwidth}}
    \caption{Бизнес-правила\label{tab:business-rules}} \\
    \hline
    \textbf{Иденти-фикатор} & \textbf{Определение правила} & \textbf{Тип правила} & \textbf{Статическое или динамическое} & \textbf{Источник} \\
    \hline \endhead
    BR-1                   & Входной файл должен иметь возможность быть скомпилированным с помощью компилятора gcc-g++ с параметром компиляции стандарта C++17 и без использования утилит по автоматизации сборки, таких как CMake                & Ограничение          & Статическое                           & Компилятор        \\
    BR-2                   & Структура входной программы не должна быть сложной (без использования оператора перехода \texttt{goto}, наследования и полиморфизма)                                                                                    & Ограничение          & Статическое                           & Анализатор        \\
    BR-3                   & Точка входа в программу должна вызывать тестируемые функции                                                                                            & Особенность          & Статическое                           & Анализатор        \\
    BR-4                   & Если исходный код программы пользователя не обладает проблемными участками, но он выбрал тестирование с подробным выводом, то в качестве результата пользователь получает, какие тестовые шаблоны были сгенерированы & Вывод                & Динамическое                          & Пользователь
\end{xltabular}
