\section{Описание функциональных требований}\label{sec:description-of-functional-requirements}

\subsection{Описание}\label{subsec:description}

Пользователь может протестировать свою программу, имея её исходный код. Пользотваель также может посмотреть результаты тестирования своей программы, и если его программа начинает некорректно работать, ему предоставляются тестовые входные данные, на которых возникла ошибка, и принцип, по которому они были сгенерированы.

\subsection{Функциональные требования}\label{subsec:functional-requirements}

\setlength{\parskip}{1.0ex}
\renewcommand{\arraystretch}{2}

\begin{xltabular}[h]{\textwidth}{XX}
    \caption{Функциональные требования\label{tab:functional-requirements}} \\
    \hline
    \textbf{Программа.ВходныеДанные:}   & \textbf{Входной файл пользователя}                                                                                                                                                                                                                                             \\
    \hspace{0.5cm}.ФайлСуществует:      & Система должна выдавать сообщение о том, что смогла успешно прочесть файл                                                                                                                                                                                                      \\
    \hspace{0.5cm}.ФайлНеСуществует:    & Система должна выдавать сообщение о том, что не смогла успешно прочесть входной файл (с указанием ошибки)                                                                                                                                                                      \\ \hline
    \textbf{Программа.Анализ:}          & \textbf{Анализ исходного кода программы}                                                                                                                                                                                                                                       \\
    \hspace{0.5cm}.АСД:                 & Система должна строить абстрактное синтаксическое дерево (AST) на основе исходного кода программы, подаваемой на вход системе                                                                                                                                                  \\
    \hspace{0.5cm}.ОшибкаСинтаксиса:    & Если исходный код программы (файл, содержащий начальную точку выполнения программы), подаваемой на вход системе, не удовлетворяет стандарту C++17 языка программирования C++, то система должна указать, в каком месте входного файла произошла ошибка синтаксического разбора \\ \hline
    \textbf{Программа.Генерация:}       & \textbf{Генерация тестовых шаблонов на основе анализа}                                                                                                                                                                                                                      \\
    \hspace{0.5cm}.ГраничныеЗначения:   & Система должна генерировать граничные значения переменных, которые могут изменяться                                                                                                                                                                                            \\
    \hspace{0.5cm}.Ветвления:           & Сгенерированные тестовые шаблоны должны полностью покрывать каждый участок кода программы                                                                                                                                                                                      \\
    \hspace{0.5cm}.Циклы:               & Сгенерированные тестовые шаблоны должны позволять проверить воможность зацикливания программы                                                                                                                                                                                  \\
    \hspace{0.5cm}.УтечкаПамяти:        & Сгенерированные тестовые шаблоны должны позволять обнаруживать участки кода, где может произойти утечка памяти                                                                                                                                                                 \\
    \hspace{0.5cm}.ИзмененнаяПрограмма: & Система должна генерировать свой код программы на основе исходного кода программы пользователя, в котором добавляются только контрольные точки, в которых можно узнать состояние программы в данный момент                                                                     \\ \hline
    \textbf{Программа.Компиляция:}      & \textbf{Компиляция изменённого системой исходного кода программы}                                                                                                                                                                                                              \\
    \hspace{0.5cm}.Компилятор:          & Система должна использовать компилятор gcc-g++ для получения исполняемого файла из измененного системой исходного кода программы                                                                                                                                               \\ \hline
    \textbf{Программа.Тестирование:}    & \textbf{Тестирование исходного кода программы}                                                                                                                                                                                                                                 \\
    \hspace{0.5cm}.НаОсновеГенерации:   & Система должна провести тестирование, используя сгенерированные шаблоны тестов и исполняемый файл, полученный при компиляции измененного системой исходного кода программы                                                                                                     \\ \hline
    \textbf{Программа.ВыходныеДанные:}  & \textbf{Получение и показ результатов тестирования}                                                                                                                                                                                                                            \\
    \hspace{0.5cm}.Ошибки:              & Система должна показать все найденные проблемные зоны исходной программы                                                                                                                                                                                                       \\
    \hspace{0.5cm}.ШаблоныТестов:       & Если в программе были найдены ошибки, система должна показать значения входных тестовых данных                                                                                                                                                                                 \\
    \hspace{0.5cm}.Объяснение:          & Если в программе были найдены ошибки, система должна объяснить, каким образом были сгенерированы входные тестовые данные, на которых программа начинает некорректно работать                                                                                                   \\
    \hspace{0.5cm}.УтечкаПамяти:        & Система должна выдавать данные о размере утекшей памяти
\end{xltabular}
