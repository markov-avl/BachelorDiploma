\section{Бизнес-требования}\label{sec:business-requirements}


\subsection{Исходные данные}\label{subsec:initial-data}


На сегодняшний день UNIT-тестирование программы для начинающих программистов остаётся одной из сложнейших задач в разработке программного обеспечения.

Система обучения написания тестовых сценариев должна уметь анализировать исходный код программы, подаваемой на вход, и выводить пользователю сгенерированные тестовые варианты входных данных, покрывающих 100\% кода, а также объяснять по какому принципу они были сгенерированы. Такая программа также должна уметь запускать тесты, которые были сформированы при анализе, чтобы наглядно указать пользователю его возможные ошибки при проектировании программы.

Конечно, невозможна такая система, гарантирующая полное покрытие кода абсолютно любой программы, поэтому система должна быть ограничена тестированием классических задач, которые обычно решают начинающие программисты на пути своего обучения. Такая система сможет наиболее эффективно и понятно донести принципы тестирования по белому ящику студентам компьютерных технологий.

Таким образом, данная программа позволит сформировать базовые знания и умения в тестировании программного обеспечения.


\subsection{Бизнес-цели}\label{subsec:business-goals}

\begin{itemize}
	\item Достигнуть показателя установок, равного 50, в течение 1 месяца после выпуска продукта;
	\item Получить удовлетворение как минимум от 80\% пользователей программного средства в течение 6 месяцев после выпуска продукта;
	\item Достигнуть показателя установок, равного 600, в течение 12 месяцев после выпуска продукта.
\end{itemize}


\subsection{SMART-анализ}\label{subsec:smart-analysis}

% тупо стащил, даже не думал ни о чем

\begin{table}[H]
	
	\label{tab:smart-analysis}
	\caption{SMART-анализ}
	\small
	\setlength{\parskip}{1.0ex}
	\renewcommand{\arraystretch}{1.5}
	
	\begin{tabular}{| p{0.2\textwidth} | p{0.50\textwidth} | p{0.2\textwidth} |}
		\hline
		\textbf{Критерий} & \textbf{Описание} & \textbf{Выполнимость} \\
		\hline \hline
		Specific (конкретные) & Все цели чётко определены & Да \\
		\hline
		Measurable (измеримые) & Все цели могут быть измеримы & Да \\
		\hline
		Achievable (достижимые) & Все цели теоретически достижимы & Да \\
		\hline
		Relevant (значимые) & Все цели важны для проекта & Да \\
		\hline
		Timebound (ограниченные по времени) & Для каждой цели указан крайний срок, за который её нужно достичь & Да \\
		\hline
	\end{tabular}
	
\end{table}

Таким образом, цели удовлетворяют всем пяти критериям.


\subsection{Видение решения}\label{subsec:solution-vision}

Решением данной проблемы может стать фреймворк, способный помочь неопытным программистам в освоении навыков тестирования программ за счёт анализа исходного кода (ветвлений, циклов и т.д.). Таким образом, такой фреймворк сможет показать программисту слабые стороны его программы на определенных входных данных и объяснить, откуда и по какому принципу были получены именно такие шаблоны тестов, чтобы обучить программиста лучше обдумывать структуру и логику его программы и в дальнейшем уметь применять полученные навыки в написании более сложных программных обеспечений.


\newpage
