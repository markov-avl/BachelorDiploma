\subsection{Глоссарий терминов}\label{subsec:glossary}

% ЗДЕСЬ НАХОДЯТСЯ ВСЕ ТЕРМИНЫ, КОТОРЫЕ УЧАВСТВУЮТ ЧУТЬ ЛИ НЕ В КАЖДОЙ ПОДСЕКЦИИ
% более "узкие" термины лучше написать в подсекции, к которому оно и принадлежит

Тестирование программного обеспечения (\textit{software testing}) -- процесс анализа программного средства и сопутствующей документации с целью выявления дефектов и повышения качества продукта.

Автоматизированное тестирование (\textit{automated testing}) -- набор разных техник, подходов и инструментальных средств, позволяющих исключить человека из выполнения некоторых задач в процессе тестирования.

% Тестовый сценарий (\textit{test case}) -- набор входных данных, условий выполнения и ожидаемых результатов, разработанный с целью проверки того или иного свойства или поведения программного средства.

% Динамическое тестирование (\textit{dynamic testing}) -- тестирование с запуском кода на исполнение.


% все определения выше взяты из software-testing (в конце)


%Тестовый сценарий высокого уровня (high level test case) -- тестовый сценарий без конкретных (уровня реализации) значений входных данных и ожидаемых результатов. Использует логические операторы, а экземпляры реальных значений еще не определены и/или доступны.

% Тестовый сценарий низкого уровня (low level test case) -- тестовый сценарий с конкретными (уровня реализации) значениями входных данных и ожидаемых результатов. Логические операторы из тестовых сценариев высокого уровня заменяются реальными значениями, которые соответствуют целям этих логических операторов.