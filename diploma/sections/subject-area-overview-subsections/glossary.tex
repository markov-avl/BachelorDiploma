\subsection{Глоссарий терминов}\label{subsec:glossary}

% ЗДЕСЬ НАХОДЯТСЯ ВСЕ ТЕРМИНЫ, КОТОРЫЕ УЧАВСТВУЮТ ЧУТЬ ЛИ НЕ В КАЖДОЙ ПОДСЕКЦИИ
% более "узкие" термины лучше написать в подсекции, к которому оно и принадлежит

Тестирование программного обеспечения (\textit{software testing}) -- процесс анализа программного средства и сопутствующей документации с целью выявления дефектов и повышения качества продукта.

Автоматизированное тестирование (\textit{automated testing}) -- набор разных техник, подходов и инструментальных средств, позволяющих исключить человека из выполнения некоторых задач в процессе тестирования.

\textbf{Тест кейс} -- это формально описанный алгоритм тестирования программы, специально созданный для определения возникновения в программе определённой ситуации, определённых выходных данных.

Тест кейсы разделяются по ожидаемому результату на позитивные и негативные:

\begin{enumerate}
    \item Позитивный тест кейс использует только корректные данные и проверяет, что приложение правильно выполнило вызываемую функцию.
    \item Негативный тест кейс оперирует как корректными так и некорректными данными (минимум 1 некорректный параметр) и ставит целью проверку исключительных ситуаций (срабатывание валидаторов), а также проверяет, что вызываемая приложением функция не выполняется при срабатывании валидатора.
\end{enumerate}

Каждый тест кейс должен иметь 3 части:
\begin{enumerate}
    \item PreConditions -- список действий, которые приводят систему к состоянию пригодному для проведения основной проверки. Либо список условий, выполнение которых говорит о том, что система находится в пригодном для проведения основного теста состояния.
    \item Test Case Description -- список действий, переводящих систему из одного состояния в другое, для получения результата, на основании которого можно сделать вывод о удовлетворении реализации, поставленным требованиям
    \item PostConditions -- список действий, переводящих систему в первоначальное состояние (\textit{initial state})
\end{enumerate}

Пример тест кейса 1:
\begin{enumerate}
    \item Проверка отображения страницы.
    \item Действие: Открыть страницу <<Вход в систему>>.
    \item Ожидаемый результат:
        \begin{enumerate}
            \item окно <<Вход в систему>> открыто;
            \item название окна -- <<Вход в систему>>;
            \item логотип компании отображается в правом верхнем углу;
            \item на форме 2 поля -- <<Имя>> и <<Пароль>>;
            \item кнопка <<Вход доступна>>;
            \item ссылка <<Забыл пароль>> -- доступна.
        \end{enumerate}
\end{enumerate}

Пример тест кейса 2:
\begin{enumerate}
    \item Название: Проверка отображения страницы.
    \item Действие: Открыть страницу <<Вход в систему>>.
    \item Проверка: Проверьте, что отображаемая страница соответствует странице на макете (и прилагаем изображение страницы <<Вход в систему>>).
\end{enumerate}

В примере 1 и 2 покрытие будет одинаковым, но вот время, которое потребуется для прохождения, будет разным.

\textbf{Чек-лист} (\textit{check-list}) -- список, содержащий ряд необходимых проверок для какой-либо работы.

В тестировании чек-лист — это список проверок для тестирования продукта. Чек-листы устроены предельно просто. Любой из них содержит перечень блоков, секций, страниц, других элементов, которые следует протестировать, например:
\begin{enumerate}
    \item Регистрация:
        \begin{enumerate}
            \item Email;
            \item Соцсети;
            \item Валидация полей.
        \end{enumerate}
    \item Авторизация
        \begin{enumerate}
            \item Пользователь активирован;
            \item Пользователь не активирован;
            \item Восстановление пароля;
            \item Валидация полей.
        \end{enumerate}
    \item Профиль
        \begin{enumerate}
            \item Изменение имени;
            \item Изменение email;
            \item Изменение пароля;
            \item Валидация полей;
            \item Подписка;
            \item Удаление аккаунта.
        \end{enumerate}
\end{enumerate}

Выполненные пункты отмечаются следующими статусами: <<Passed>> (у-спешно), <<Failed>> (провалено), <<Blocked>> (блокирующий), <<Skipped>> (пропущен), <<Not run>> (не запускался).

Преимущества использования чек-листов: 
\begin{enumerate}
    \item улучшить представление о системе в целом, видеть статус ее готовности;
    \item понимать объем проделанной и предстоящей работы по тестированию;
    \item не повторяться в проверках и не упустить ничего важного в процессе тестирования.
\end{enumerate}

\textbf{Баг или дефект репорт} (\textit{bug}) -- это документ, описывающий ситуацию или последовательность действий приведшую к некорректной работе объекта тестирования, с указанием причин и ожидаемого результата.

Баг репорт (\textit{but report}) -- это технический документ, причём язык описания проблемы должен быть техническим. Должна использоваться правильная терминология при использовании названий элементов пользовательского интерфейса (editbox, listbox, combobox, link, text area, button, menu, popup menu, title bar, system tray и т.д.), действий пользователя (click link, press the button, select menu item и т.д.) и полученных результатах (window is opened, error message is displayed, system crashed и т.д.).

Структура баг репорта:
\begin{enumerate}
    \item \textbf{Шапка}
        \begin{itemize}
            \item Короткое описание (\textit{Summary}) -- короткое описание проблемы, явно указывающее на причину и тип ошибочной ситуации.
            \item Проект (\textit{Project}) -- название тестируемого проекта.
            \item Компонент приложения (\textit{Component}) -- название части или функции тестируемого продукта.
            \item Номер версии (\textit{Version}) -- версия на которой была найдена ошибка.
            \item Серьезность (\textit{Severity}) -- наиболее распространена пятиуровневая система градации серьезности дефекта:
                \begin{enumerate}
                    \item S1 -- Блокирующий (\textit{Blocker}).
                    \item S2 -- Критический (\textit{Critical}).
                    \item S3 -- Значительный (\textit{Major}).
                    \item S4 -- Незначительный (\textit{Minor}).
                    \item S5 -- Тривиальный (\textit{Trivial}).
                \end{enumerate}
            \item Серьезность (\text{Severity}) -- наиболее распространена пятиуровневая система градации серьезности дефекта:
                \begin{enumerate}
                    \item S1 -- Блокирующий (\textit{Blocker}).
                    \item S2 -- Критический (\textit{Critical}).
                    \item S3 -- Значительный (\textit{Major}).
                    \item S4 -- Незначительный (\textit{Minor}).
                    \item S5 -- Тривиальный (\textit{Trivial}).
                \end{enumerate}
            \item Приоритет (\textit{Priority}) -- приоритет бага:
                \begin{enumerate}
                    \item P1 -- Высокий (\textit{High}).
                    \item P2 -- Средний (\textit{Medium}).
                    \item P3 -- Низкий (\textit{Low}).
                \end{enumerate}
            \item Статус (\textit{Status}) -- cтатус бага. Зависит от используемой процедуры и жизненного цикла бага (\textit{bug workflow and life cycle}).
            \item Автор (\textit{Author}) -- cоздатель баг-репорта.
            \item Назначен на (\textit{Assigned To}) -- имя сотрудника, назначенного на решение проблемы.
        \end{itemize}
    \item \textbf{Окружение} \\
         ОС / Сервис Пак и т.д. / Браузер + версия -- Информация об окружении, на котором был найден баг: операционная система, сервис пак, для WEB тестирования -- имя и версия браузера и т.д.
    \item \textbf{Описание}
        \begin{itemize}
            \item Шаги воспроизведения (\textit{Steps to Reproduce}) -- шаги, по которым можно легко воспроизвести ситуацию, приведшую к ошибке.
            \item Фактический Результат (\textit{Result}) -- результат, полученный после прохождения шагов к воспроизведению
            \item Ожидаемый результат (\textit{Expected Result})	-- ожидаемый правильный результат.
        \end{itemize}
    \item \textbf{Дополнения}
        \begin{itemize}
            \item Прикрепленный файл (\text{Attachment}) -- файл с логами, скриншот или любой другой документ, который может помочь прояснить причину ошибки или указать на способ решения проблемы.
        \end{itemize}
\end{enumerate}

Серьезность (\text{Severity}) -- это атрибут, характеризующий влияние дефекта на работоспособность приложения.

Приоритет (\text{Priority}) -- это атрибут, указывающий на очередность выполнения задачи или устранения дефекта. Можно сказать, что это инструмент менеджера по планированию работ. Чем выше приоритет, тем быстрее нужно исправить дефект.