\subsection{Виды и методы тестирования программного обеспечения}\label{subsec:testing-methods}

Методов тестирования очень много, они различаются как по способу тестирования, так и виду тестирования. Существуют методы тестирования не использующие код программы, такие как:
\begin{enumerate}
    \item метод черного ящика;
    \item тестирование требований.
\end{enumerate}

Поскольку такие методы тестирования не используют код программы они не зависят от реализации, вследствие чего при таком тестирование возможно написание тестовых сценариев за долго до реализации. Также из-за отсутствия реализации появляется проблема, поскольку такое тестирование невозможно автоматизировать, его всегда приходится выполнять вручную.

Метод черного ящика -- тестирование, основанное на анализе функциональной или нефункциональной спецификации компонента или системы без знания внутренней структуры.

Тестирование требований -- подход к тестированию, при котором тестовые сценарии разрабатываются на основе целей и условий тестирования, вытекающих из требований; то есть тесты, проверяющие определенный функционал или оценивающие нефункциональные атрибуты системы, такие как надежность или практичность.

Также существуют методы тестирования, которые зависят от реализации, точнее невозможно начать подготовку тестовых сценариев пока не будет готова реализация ПО. Пример таких методов:
\begin{enumerate}
    \item метод белого ящика;
    \item нагрузочное тестирование;
    \item тестирование удобства пользования.
\end{enumerate}

Метод белого ящика -- процедура разработки или выбора тестовых сценариев на основании анализа внутренней структуры компонента или системы.

Нагрузочное тестирование -- вид тестирования, оценивающий систему или компонент на граничных значениях рабочих нагрузок или за их пределами, или же в состоянии ограниченных ресурсов, таких как память или доступ к серверу.

Тестирование удобства пользования -- тестирование с целью определения степени понятности, легкости в изучении и использовании, привлекательности программного продукта для пользователя при условии использования в заданных условиях эксплуатации.

Методы имеют разные цели и разные подходы, исходя из того, что видов и методов тестирования много, в данной работе выбран метод белого ящика, так как можно использовать код программы, который строго структурирован, легок в освоении и широко используется для тестирования.