\subsubsection{Классификация по доступу к коду программы}\label{subsubsec:by-access}

\textbf{Тестирование чёрного ящика} -- техника тестирования, основанная на работе исключительно с внешними интерфейсами тестируемой системы.

Согласно ISTQB (организация, занимающаяся вопросами развития сферы тестирования ПО), данная техника предполагает как функциональное, так и нефункциональное тестирование без знания внутренного устройства системы или его компонента.\cite{isqtb}

Целью данной техники является поиск ошибок в следующих категорях:
\begin{itemize}
    \item ошибки пользовательского интерфейса;
    \item ошибки поведения системы;
    \item проблемы производительности системы;
    \item нереализованные или некорректно работающие функции;
    \item неверно построенная реализация системы.
\end{itemize}

Используя данный подход, мы больше концетрируемся на том, что программа делает, а не на том, как она это делает.

\textbf{Тестирование белого ящика} -- техника тестирования программного обеспечения, которая предполагает, что внутренняя устройство системы известно тестировщику. Мы выбираем входные значения, основываясь на знании кода, который будет их обрабатывать. Точно так же мы знаем, каким должен быть результат этой обработки. Знание всех особенностей тестируемой программы и ее реализации -- обязательны для этой техники. Тестирование белого ящика -- углубление во внутренне устройство системы, за пределы ее внешних интерфейсов. 

Согласно ISTQB, данная техника основывается на анализе внутренней структуры компонента или системы.

В процессе моделирования тестовых сценариев стараются покрыть код. Покрытие кода (\textit{code coverage}) -- мера, используемая при тестировании программного обеспечения. Она показывает процент исходного кода программы, который был выполнен в процессе тестирования. Существует несколько различных способов измерения покрытия, основные из них:
\begin{enumerate}
    \item покрытие операторов -- каждая ли строка исходного кода была выполнена и протестирована;
    \item покрытие условий -- каждая ли точка решения (вычисления истинно ли или ложно выражение) была выполнена и протестирована;
    \item покрытие путей -- все ли возможные пути через заданную часть кода были выполнены и протестированы;
    \item покрытие функций -- каждая ли функция программы была выполнена;
    \item покрытие вход/выход -- все ли вызовы функций и возвраты из них были выполнены;
    \item покрытие значений параметров -- все ли типовые и граничные значения параметров были проверены.
\end{enumerate}

\textbf{Тестирование серого ящика} -- техника тестирования, предполагающая неполные знание внутреннего устройства системы у разработчика. Чтобы выполнить подобный вид тестов, не нужно иметь доступ к исходному коду ПО.\cite{quality-lab}

Согласно ISTQB, данная техника основывается на знаниях алгоритмов, архитектуры, внутренних состояний или других высокоуровневых описаний поведения системы.

Данный метод -- это комбинация двух предыдущих подходов (тестирование чёрного и белого ящиков). Специалист стремится найти все проблемы функционирования и ошибки в коде. На этой стадии тестировщик может реализовать сквозной тест.

Для запуска тестовых случаев во время тестирования серого ящика не обязателен доступ к коду. Каждый тест базируется на знании поведения программы. Это хороший подход к реализации функционального тестирования. Однако, это не будет успешным без реализации более глубоких методов, таких как тестирование белого и черного ящиков.\cite{careerist}

% сделать сравнение ящиков в виде таблиц