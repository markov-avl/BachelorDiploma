\subsubsection{Классификация по запуску кода программы}\label{subsubsec:by-starting}

Тестирование не всегда предполагает взаимодействие с работающей системой. Многие ошибки можно обнаружить и до запуска кода программы, например, по его исходному коду. Этот процесс и называется статическим тестированием (\textit{static testing})

В рамках использования этого подхода тестированию могут подвергаться любые формы документации (требования, схемы, описания системы и т.д.) и исходный код программы.

Статическое тестирование часто автоматизируют. Например, разработчики часто используют автоматические средства проверки синтаксиса программного кода, которые предупреждают о возможных допущенных ошибках.

\begin{description}
    \item[Пример статического тестирования:]\leavevmode
        \begin{lstlisting}[caption={В данном случае операция очистки выделенной памяти (строка 5) описана после выхода из функции (строка 4), поэтому никогда не будет исполнена.}]
SomeObject *object = new SomeObject();
if (!object->objectMethod())
{
    return false;
    <@\textcolor{red}{delete}@> object;
}
        \end{lstlisting}
\end{description}

Также существует динамическое тестирование -- тестирование с запуском кода на исполнение. Запускаться на исполнение может как код всего приложения целиком (системное тестирование), так и код нескольких взаимосвязанных частей (интеграционное тестирование), отдельных частей (модульное или компонентное тестирование) и даже отдельные участки кода. Основная идея этого вида тестирования состоит в том, что проверяется реальное поведение (части) приложения.

Только при таком тестировании возможна проверка внешних параметров работы программы: загрузка процессора, использование памяти, время отклика и т.д. -- то есть, ее производительность.

% последний абзац украл отсюда: https://qalight.ua/ru/baza-znaniy/staticheskoe-i-dinamicheskoe-testirovanie