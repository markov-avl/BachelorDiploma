\subsubsection{Классификация по запуску кода программы}\label{subsubsec:by-starting}

Тестирование не всегда предполагает взаимодействие с работающей системой. Многие ошибки можно обнаружить и до запуска кода программы, например, по его исходному коду. Этот процесс и называется статическим тестированием (\textit{static testing})

В рамках использования этого подхода тестированию могут подвергаться любые формы документации (требования, схемы, описания системы и т.д.) и исходный код программы.

\lstset{language=C++,caption={Пример статического тестирования}}
\begin{lstlisting}
    SomeObject *object = new SomeObject();
    if (!object->objectMethod())
    {
        return false;
        delete t;
    }
\end{lstlisting}

Также такое тестирование возможно автоматизировать. Например, можно использовать автоматические средства проверки синтаксиса программного кода, которые предупреждают разработчика о возможной допущенной ошибки.

% Далеко не всякое тестирование предполагает взаимодействие с работающим приложением.
% Потому в рамках данной классификации выделяют:
%     1. Статическое тестирование (static testing) — тестирование без запуска кода на исполнение. В рамках этого подхода тестированию могут подвергаться:
%     • Документы (требования, тест-кейсы, описания архитектуры приложения, схемы баз данных и т.д.).
%     • Графические прототипы (например, эскизы пользовательского интерфейса).
%     • Код приложения. Код приложения также можно проверять с использованием техник тестирования на основе структур кода.
%     • Параметры (настройки) среды исполнения приложения.
%     • Подготовленные тестовые данные.
%     2. Динамическое тестирование — тестирование с запуском кода на исполнение. Запускаться на исполнение может как код всего приложения целиком (системное тестирование), так и код нескольких взаимосвязанных частей (интеграционное тестирование), отдельных частей (модульное или компонентное тестирование) и даже отдельные участки кода. Основная идея этого вида тестирования состоит в том, что проверяется реальное поведение (части) приложения.