\subsubsection{Тестирование классами эквивалентности}\label{subsubsec:equivalence-classes}

Класс эквивалентности (\textit{equivalence class}) -- одно или несколько значений ввода, к которым программное обеспечение применяет одинаковую логику.

Два теста можно считать эквивалентными, в случае когда:
\begin{enumerate}
    \item они проверяют одну и ту же часть системы (функцию, модуль);
    \item один тест находит ошибку, то и другой, скорее всего, найдет ошибку и наоборот (если один не находит ошибку – второй также не находит);
    \item они используют сходные наборы входных данных;
    \item чтобы выполнить тесты, необходимо совершить одни и те же операции;
    \item в результате проведения тестов получаем одинаковые выходные данные и система находится в одном и том же состоянии;
    \item срабатывает один и тот же блок обработки ошибки;
    \item не срабатывает блок обработки ошибки.
\end{enumerate}

Шаги применения техники разделения на классы эквивалентности, следующие:
\begin{enumerate}
    \item определить классы эквивалентности;
    \item выбрать представителя каждого класса;
    \item выполнить тесты.
\end{enumerate}

Пример выполнения: есть поле ввода с диапазоном допустимых значений от 1 до 100. Если проверять каждое значение из этого диапазона, это будет очень долго и не эффективно. В данном примере можно выделить 2 класса эквивалентности:
\begin{enumerate}
    \item Допустимые значения(от 1 до 100).
    \item Недопустимые значения:
        \begin{enumerate}
            \item от $-\infty$ до 0;
            \item от 101 до $+\infty$;
            \item буквы;
            \item специальные символы (@-\#!).
        \end{enumerate}
\end{enumerate}

Используя классы эквивалентности можно протестировать поле ввода минимум из 5 тестов.