\subsubsection{Тестирование граничных значений}\label{subsubsec:boundary-testing}

Граничные значения -- это те места, в которых один класс эквивалентности переходит в другой.

На каждой границе диапазона следует проверить по три значения:
\begin{enumerate}
    \item граничное значение;
    \item значение перед границей;
    \item значение после границы.
\end{enumerate}

Цель этой техники -- найти ошибки, связанные с граничными значениями.

Алгоритм использования техники граничных значений:
\begin{enumerate}
    \item выделить классы эквивалентности;
    \item определить граничные значения этих классов;
    \item нужно понять, к какому классу будет относиться каждая граница;
    \item нужно провести тесты по проверке значения до границы, на границе и сразу после границы.
\end{enumerate}

Пример использования:
\begin{enumerate}
    \item Условие -- в поля ввода можно внести только целые числа от 0 до 10 000.
    \item Определяемся с существующими границами -- так как в условии все значения от 0 до 10 000 приведут к одному и тому же результату, то границы две: нижняя и верхняя.
    \item\label{item:first-boundary} Первое граничное значение -- 0.
    \item\label{item:second-boundary} Второе граничное значение -- 10 000.
    \item Добавляем к ним (пункты \ref{item:first-boundary}, \ref{item:second-boundary}), стоящие рядом значения:
        \begin{enumerate}
            \item -1, 0, 1 (для пункта \ref{item:first-boundary}).
            \item 9 999, 10 000, 10 001 (для пункта \ref{item:second-boundary}).
        \end{enumerate}
\end{enumerate}