\subsection{Дизайн тестовых сценариев}\label{subsec:test-design}

Тест дизайн -- один из первоначальных этапов тестирования программного обеспечения, этап планирования и проектирования тестов. Тест дизайн представляет собой продумывание и написание тестовых случаев, в соответствии с требованиями проекта, критериями качества будущего продукта и финальными целями тестирования.

Цель тест дизайна заключается в обеспечении покрытия функционала приложения тестами:
\begin{enumerate}
    \item Тесты должны покрывать весь функционал.
    \item Тестов должно быть минимально достаточно.
\end{enumerate}

Задачи тест дизайна:
\begin{enumerate}
    \item Проанализировать требования к продукту.
    \item Оценить риски возможные при использовании продукта.
    \item Написать достаточное минимальное количество тестов.
    \item Разграничить тесты на приемочные, критические, расширенные.
\end{enumerate}

Существуют несколько техник тест дизайна. Рассмотрим следующие:
\begin{enumerate}
    \item Тестирование классами эквивалентности (\textit{Equivalence Class Testing}).
    \item Тестирование граничных значений (\textit{Boundary Value Testing}).
    \item Таблица принятия решений (\textit{Decision Table Testing}).
\end{enumerate}

\subsubsection{Тестирование классами эквивалентности}\label{subsubsec:equivalence-classes}

Класс эквивалентности (\textit{equivalence class}) -- одно или несколько значений ввода, к которым программное обеспечение применяет одинаковую логику.

Два теста можно считать эквивалентными, в случае когда:
\begin{enumerate}
    \item они проверяют одну и ту же часть системы (функцию, модуль);
    \item один тест находит ошибку, то и другой, скорее всего, найдет ошибку и наоборот (если один не находит ошибку – второй также не находит);
    \item они используют сходные наборы входных данных;
    \item чтобы выполнить тесты, необходимо совершить одни и те же операции;
    \item в результате проведения тестов получаем одинаковые выходные данные и система находится в одном и том же состоянии;
    \item срабатывает один и тот же блок обработки ошибки;
    \item не срабатывает блок обработки ошибки.
\end{enumerate}

Шаги применения техники разделения на классы эквивалентности, следующие:
\begin{enumerate}
    \item определить классы эквивалентности;
    \item выбрать представителя каждого класса;
    \item выполнить тесты.
\end{enumerate}

Пример выполнения: есть поле ввода с диапазоном допустимых значений от 1 до 100. Если проверять каждое значение из этого диапазона, это будет очень долго и не эффективно. В данном примере можно выделить 2 класса эквивалентности:
\begin{enumerate}
    \item Допустимые значения(от 1 до 100).
    \item Недопустимые значения:
        \begin{enumerate}
            \item от $-\infty$ до 0;
            \item от 101 до $+\infty$;
            \item буквы;
            \item специальные символы (@-\#!).
        \end{enumerate}
\end{enumerate}

Используя классы эквивалентности можно протестировать поле ввода минимум из 5 тестов.
\subsubsection{Тестирование граничных значений}\label{subsubsec:boundary-testing}

Граничные значения -- это те места, в которых один класс эквивалентности переходит в другой.

На каждой границе диапазона следует проверить по три значения:
\begin{enumerate}
    \item граничное значение;
    \item значение перед границей;
    \item значение после границы.
\end{enumerate}

Цель этой техники -- найти ошибки, связанные с граничными значениями.

Алгоритм использования техники граничных значений:
\begin{enumerate}
    \item выделить классы эквивалентности;
    \item определить граничные значения этих классов;
    \item нужно понять, к какому классу будет относиться каждая граница;
    \item нужно провести тесты по проверке значения до границы, на границе и сразу после границы.
\end{enumerate}

Пример использования:
\begin{enumerate}
    \item Условие -- в поля ввода можно внести только целые числа от 0 до 10 000.
    \item Определяемся с существующими границами -- так как в условии все значения от 0 до 10 000 приведут к одному и тому же результату, то границы две: нижняя и верхняя.
    \item\label{item:first-boundary} Первое граничное значение -- 0.
    \item\label{item:second-boundary} Второе граничное значение -- 10 000.
    \item Добавляем к ним (пункты \ref{item:first-boundary}, \ref{item:second-boundary}), стоящие рядом значения:
        \begin{enumerate}
            \item -1, 0, 1 (для пункта \ref{item:first-boundary}).
            \item 9 999, 10 000, 10 001 (для пункта \ref{item:second-boundary}).
        \end{enumerate}
\end{enumerate}
\subsection{Таблица принятия решений}\label{subsec:Decision-Table}

\textbf{Таблица решений} (\textit{decision table}) -- способ компактного представления модели со сложной логикой; инструмент для упорядочения сложных бизнес требований, которые должны быть реализованы в продукте. Это взаимосвязь между множеством условий и действий. \cite{education-wiki}

Таблица принятия решений, как правило, разделяется на 4 квадранта:
\begin{enumerate}
    \item Условия -- список возможных условий.
    \item Варианты выполнения действий -- комбинация из выполнения и/или невыполнения условий этого списка.
    \item Действия -- список возможных действий.
    \item Необходимость действий -- указание надо или не надо выполнять соответствующее действие для каждой из комбинаций условий.
\end{enumerate}

Рассмотрим таблицу принятия решений на примере страницы регистрации нового пользователя:
\begin{enumerate}
    \item Используем понятия <<корректные>> и <<некорректные>> данные.
    \item Регистрация проходит успешно, если оба поля заполнены корректно.
    \item Если поля заполняются некорректными данными, то система должна выдать ошибку: <<Введены невалидные данные>>.
\end{enumerate}

\setlength{\parskip}{1.0ex}
\renewcommand{\arraystretch}{2}

\begin{xltabular}[h]{\textwidth}{|p{0.25 \textwidth}|C|C|C|C|}
    \caption{Таблица принятия решений -- условия\label{tab:making-decisions-conditions}} \\
    \hline
    \textbf{Условие}                         & \textbf{Значение 1} & \textbf{Значение 2} & \textbf{Значение 3} & \textbf{Значение 4} \\ \hline \endhead
    Ввод корректных данных в поле E-mail     & +                   & -                   & +                   & -                   \\ \hline
    Ввод корректных данных в поле Password   & +                   & -                   & -                   & +                   \\ \hline
    Ввод некорректных данных в поле E-mail   & -                   & +                   & -                   & +                   \\ \hline
    Ввод некорректных данных в поле Password & -                   & +                   & +                   & -                   \\ \hline
\end{xltabular}

\begin{xltabular}[h]{\textwidth}{|p{0.25 \textwidth}|C|C|C|C|}
    \caption{Таблица принятия решений -- действия\label{tab:making-decisions-actions}} \\
    \hline
    \textbf{Действия}                             & \textbf{Значение 1} & \textbf{Значение 2} & \textbf{Значение 3} & \textbf{Значение 4} \\ \hline \endhead
    Регистрация прошла успешно                    & +                   & -                   & -                   & -                   \\ \hline
    Выдаётся ошибка <<Введены невалидные данные>> & -                   & +                   & +                   & +                   \\ \hline
\end{xltabular}

Значения 2, 3, 4 (таблица \ref{tab:making-decisions-actions}) приводят к одному и тому же результату с разными входными значениями (таблица \ref{tab:making-decisions-conditions}).