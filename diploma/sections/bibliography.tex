\addcontentsline{toc}{section}{Список литературы}

\begin{thebibliography}{7}
    \bibitem{software-testing} Куликов С.С. Тестирование программного обеспечения. Базовый курс. 3-е издание [Книга] / Куликов С.С. -- EPAM Systems, 2021 -- 298 c.
    \bibitem{dot-com} Савин Р. Тестирование Дот Ком, или Пособие по жестокому обращению с багами в интернет-стартапах [Книга] / Савин Р. -- Дело, 2007 -- 312 с.
    \bibitem{google-testing} Уиттакер Дж. Как тестируют в Google [Книга] / Уиттакер Дж., Арбон Дж., Каролло Дж. -- СПб.: Питер, 2014 -- 320 с.
    \bibitem{fast-testing} Капбертсон Р. Быстрое тестирование [Книга] / Капбертсон Р., Браун К., Кобб Г. -- Издательский дом <<Вильямс>>, 2002 -- 384 с.
    \bibitem{wiki-testing} Wikipedia: / Статья [Электронный ресурс] URL: https://ru.wikipedia.org/wiki/Тестирование\_программного\_обеспечения (Дата обращения 09.11.21)
    \bibitem{qalight} QALight: / Статья [Электронный ресурс] URL: https://qalight.ua/ru/baza-znaniy/staticheskoe-i-dinamicheskoe-testirovanie
    \bibitem{isqtb} ISQTB: / Документация [Электронный ресурс] URL: https://www.rstqb.org/ru/istqb-downloads.html (Дата обращения 07.12.21)
    \bibitem{quality-lab} Лаборатория Качества: / Статья [Электронный ресурс] URL: https://quality-lab.ru/blog/key-principles-of-gray-box-testing/ (Дата обращения 07.12.21)
    \bibitem{careerist} Careerist: / Статья [Электронный ресурс] URL: https://www.careerist.com/ru-insights/belyy-seryy-i-chernyy-yashchik (Дата обращения 07.12.21)
    \bibitem{aprozorova} Апрозорова: / Статья [Электронный ресурус] URL: http://aprozorova.blogspot.com/2017/10/blog-post\_11.html (Дата обращения 13.12.21)
    \bibitem{blackbox-testing} Бейзер Б. Тестирование чёрного ящика. Технология функционального тестирования программного обеспечения и систем [Книга] / Бейзер Б. -- СПб.: Питер, 2004 -- 318 с.
\end{thebibliography}

\newpage