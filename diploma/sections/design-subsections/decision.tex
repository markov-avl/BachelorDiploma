\subsection{Таблица принятия решений}\label{subsec:Decision-Table}

\textbf{Таблица решений} (\textit{decision table}) -- способ компактного представления модели со сложной логикой; инструмент для упорядочения сложных бизнес требований, которые должны быть реализованы в продукте. Это взаимосвязь между множеством условий и действий. \cite{education-wiki}

Таблица принятия решений, как правило, разделяется на 4 квадранта:
\begin{enumerate}
    \item Условия -- список возможных условий.
    \item Варианты выполнения действий -- комбинация из выполнения и/или невыполнения условий этого списка.
    \item Действия -- список возможных действий.
    \item Необходимость действий -- указание надо или не надо выполнять соответствующее действие для каждой из комбинаций условий.
\end{enumerate}

Рассмотрим таблицу принятия решений на примере страницы регистрации нового пользователя:
\begin{enumerate}
    \item Используем понятия <<корректные>> и <<некорректные>> данные.
    \item Регистрация проходит успешно, если оба поля заполнены корректно.
    \item Если поля заполняются некорректными данными, то система должна выдать ошибку: <<Введены невалидные данные>>.
\end{enumerate}

\setlength{\parskip}{1.0ex}
\renewcommand{\arraystretch}{2}

\begin{xltabular}[h]{\textwidth}{|p{0.25 \textwidth}|C|C|C|C|}
    \caption{Таблица принятия решений -- условия\label{tab:making-decisions-conditions}} \\
    \hline
    \textbf{Условие}                         & \textbf{Значение 1} & \textbf{Значение 2} & \textbf{Значение 3} & \textbf{Значение 4} \\ \hline \endhead
    Ввод корректных данных в поле E-mail     & +                   & -                   & +                   & -                   \\ \hline
    Ввод корректных данных в поле Password   & +                   & -                   & -                   & +                   \\ \hline
    Ввод некорректных данных в поле E-mail   & -                   & +                   & -                   & +                   \\ \hline
    Ввод некорректных данных в поле Password & -                   & +                   & +                   & -                   \\ \hline
\end{xltabular}

\begin{xltabular}[h]{\textwidth}{|p{0.25 \textwidth}|C|C|C|C|}
    \caption{Таблица принятия решений -- действия\label{tab:making-decisions-actions}} \\
    \hline
    \textbf{Действия}                             & \textbf{Значение 1} & \textbf{Значение 2} & \textbf{Значение 3} & \textbf{Значение 4} \\ \hline \endhead
    Регистрация прошла успешно                    & +                   & -                   & -                   & -                   \\ \hline
    Выдаётся ошибка <<Введены невалидные данные>> & -                   & +                   & +                   & +                   \\ \hline
\end{xltabular}

Значения 2, 3, 4 (таблица \ref{tab:making-decisions-actions}) приводят к одному и тому же результату с разными входными значениями (таблица \ref{tab:making-decisions-conditions}).