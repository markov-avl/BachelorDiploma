\subsection{Тестирование классами эквивалентности}\label{subsec:equivalence-classes}

\textbf{Класс эквивалентности} (\textit{equivalence class}) -- одно или несколько значений ввода, к которым программное обеспечение применяет одинаковую логику. \cite{qaevolution}

Два теста можно считать эквивалентными, в случае когда:
\begin{enumerate}
    \item они проверяют одну и ту же часть системы (функцию, модуль);
    \item один тест находит ошибку, то и другой, скорее всего, найдет ошибку и наоборот (если один не находит ошибку -- второй также не находит);
    \item они используют сходные наборы входных данных;
    \item чтобы выполнить тесты, необходимо совершить одни и те же операции;
    \item в результате проведения тестов получаем одинаковые выходные данные и система находится в одном и том же состоянии;
    \item срабатывает один и тот же блок обработки ошибки;
    \item не срабатывает блок обработки ошибки.
\end{enumerate}

Шаги применения техники разделения на классы эквивалентности, следующие:
\begin{enumerate}
    \item определить классы эквивалентности;
    \item выбрать представителя каждого класса;
    \item выполнить тесты.
\end{enumerate}

Пример выполнения: есть поле ввода с диапазоном допустимых значений от 1 до 100. Если проверять каждое значение из этого диапазона, это будет очень долго и не эффективно. В данном примере можно выделить 2 класса эквивалентности (таблица \ref{table:equivalence-example}).

\begin{table}[H]
    \caption{Пример классов эквивалентности}
    \label{table:equivalence-example}
    \setlength{\parskip}{1.0ex}
    \renewcommand{\arraystretch}{1.5}
    \renewcommand{\tabularxcolumn}[1]{m{#1}}
    \begin{tabularx}{\textwidth}{|X|X|}
        \hline
        \textbf{Допустимые значения}                    & от 1 до 100                 \\ \hline
        \multirow{4}{*}{\textbf{Недопустимые значения}} & от $-\infty$ до 0           \\ \cline{2-2} 
                                                        & от 101 до $+\infty$         \\ \cline{2-2} 
                                                        & буквы                       \\ \cline{2-2} 
                                                        & специальные символы (@-\#!) \\ \hline
    \end{tabularx}
\end{table}

Используя классы эквивалентности в данном примере (таблица \ref{table:equivalence-example}) можно протестировать поле ввода минимум из 5 тестов.