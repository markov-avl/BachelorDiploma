\section*{Введение}\label{sec:introduction}
\addcontentsline{toc}{section}{Введение}

На сегодняшний день тестирование остаётся одним из сложнейших этапов разработки программного обеспечения, порой даже более сложным, чем написание кода программы.

Существует множество методов тестирования программного обеспечения, отличающихся подходом, промежуточной целью и многими другими признаками. Нередко начинающие программисты допускают в своём коде серьёзные ошибки, приводящие к непредсказуемому поведению программы, и даже не подозревают об их существовании. По б\'{о}льшей мере это происходит по вине разработчика -- из-за недооценённости тестирования или же использования неправильного подхода к составлению тестов для своей программы.

Целью данной работы является создание программного обеспечения, позволяющего автоматически провести тестирование программы по некоторым определенным показателям, используя анализ исходного кода и подробно описывая весь ход.

Для достижения поставленной цели необходимо решить следующие задачи:
\begin{enumerate}
    \item Провести обзор предметной области <<Автоматизированное тестирование по белому ящику>>.
    \item Описать задачи, решаемые в предметной области <<Автоматизированное тестирование по белому ящику>>.
    \item Составить проект программного средства.
    \item Реализовать анализатор исходного кода программы, подаваемой на вход. % для Андрей
    \item Реализовать генератор тестовых шаблонов для программы, подаваемой на вход. % для Леонида
    \item Реализовать программное средство.
\end{enumerate}

\newpage