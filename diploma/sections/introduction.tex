\section*{Введение}\label{sec:introduction}
\addcontentsline{toc}{section}{Введение}

Главным заданием тестировщика программного обеспечения является поиск и документирование ошибок с последующим процессом их устранения для дальнейшей более качественной работы тестируемого продукта.

Тестирование неотъемлемая часть программного обеспечения (ПО). Каждое ПО даже учебное необходимо подкреплять тестовыми сценариями. Учебные задачи – задачи проектируемы в рамках учебы в ВУЗ. Чаще всего у студентов возникает проблема в создании тестов, поскольку не имеют должной подготовки.

В данной выпускной квалификационной работе описывается создание программного продукта, способствующего созданию тестовых сценариев. Целью выпускной квалификационной работы является создание приложения для создания тестовых сценариев.

Для достижения поставленной цели необходимо решить следующие задачи:
\begin{enumerate}
    \item Выполнить обзор предметной области <<Обучающая система  основам тестирования программного обеспечения>>, существующие решения.
    \item Описать задачи, решаемые в предметной области <<Обучающая система  основам тестирования программного обеспечения>>.
    \item Составить проект программного средства.
    \item Реализовать программное средство.
\end{enumerate}

\newpage