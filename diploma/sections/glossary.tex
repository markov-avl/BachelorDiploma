\section{Глоссарий терминов}\label{sec:glossary}

% ЗДЕСЬ НАХОДЯТСЯ ВСЕ ТЕРМИНЫ, КОТОРЫЕ УЧАВСТВУЮТ ЧУТЬ ЛИ НЕ В КАЖДОЙ ПОДСЕКЦИИ
% более "узкие" термины лучше написать в подсекции, к которому оно и принадлежит

\textbf{Тестирование программного обеспечения} (\textit{software testing}) -- процесс анализа программного средства и сопутствующей документации с целью выявления дефектов и повышения качества продукта. \cite{software-testing}

\textbf{Валидатор} (\textit{validator}) -- компьютерная программа, которая проверяет соответствие какого-либо потока данных определённому формату. \cite{software-testing}


\subsection{Тест кейсы}


\textbf{Тест кейс} (\textit{test case}) -- это формально описанный алгоритм тестирования программы, специально созданный для определения возникновения в программе определённой ситуации, определённых выходных данных. \cite{dot-com}

Тест кейсы разделяются по ожидаемому результату на позитивные и негативные \cite{dot-com}:
\begin{enumerate}
    \item \textbf{Позитивный тест кейс} использует только корректные данные и проверяет, что приложение правильно выполнило вызываемую функцию.
    \item \textbf{Негативный тест кейс} оперирует как корректными, так и некорректными данными (минимум 1 некорректный параметр) и ставит целью проверку исключительных ситуаций (срабатывание валидаторов), а также проверяет, что вызываемая приложением функция не выполняется при срабатывании валидатора.
\end{enumerate}

Каждый тест кейс должен иметь 3 части \cite{dot-com}:
\begin{enumerate}
    \item \textbf{Предварительные условия} (\textit{preconditions}) -- список действий, которые приводят систему к состоянию пригодному для проведения основной проверки. Либо список условий, выполнение которых говорит о том, что система находится в пригодном для проведения основного теста состояния.
    \item \textbf{Описание тест кейса} (\textit{test case description}) -- список действий, переводящих систему из одного состояния в другое, для получения результата, на основании которого можно сделать вывод о удовлетворении реализации, поставленным требованиям.
    \item \textbf{Постусловия} (\textit{postconditions}) -- список действий, переводящих систему в первоначальное состояние (\textit{initial state}).
\end{enumerate}

\begin{table}[H]
    \caption{Пример тест кейса}
    \label{table:test-case-example}
    \setlength{\parskip}{1.0ex}
    \renewcommand{\arraystretch}{1.5}
    \renewcommand{\tabularxcolumn}[1]{m{#1}}
    \begin{tabularx}{\textwidth}{|X|X|}
        \hline
        \textbf{Предварительные условия}      & Проверка отображения страницы                       \\ \hline
        \textbf{Описание тест кейса}          & Открытие страницы <<Вход в систему>>                \\ \hline
        \multirow{8}{*}{\textbf{Постусловия}} & Окно <<Вход в систему>> открыто                     \\ \cline{2-2} 
                                              & Названием окна является <<Вход в систему>>          \\ \cline{2-2} 
                                              & Логотип компании отображается в правом верхнем углу \\ \cline{2-2} 
                                              & На форме 2 поля: <<Имя>> и <<Пароль>>               \\ \cline{2-2} 
                                              & Есть кнопка <<Войти>>                               \\ \cline{2-2} 
                                              & Доступна ссылка <<Забыл пароль>>                    \\ \hline
    \end{tabularx}
\end{table}


\subsection{Чек-листы}


\textbf{Чек-лист} (\textit{check-list}) -- список, содержащий ряд необходимых проверок для какой-либо работы \cite{dot-com}.

В тестировании чек-лист -- это список проверок для тестирования продукта. Чек-листы устроены предельно просто. Любой из них содержит перечень блоков, секций, страниц, других элементов, которые следует протестировать \cite{dot-com}, например:
\begin{enumerate}[leftmargin=1cm]
    \item Регистрация:
        \begin{itemize}
            \item имя;
            \item e-mail;
            \item пароль;
            \item валидация полей.
        \end{itemize}
    \item Авторизация:
        \begin{itemize}
            \item пользователь активирован;
            \item пользователь не активирован;
            \item восстановление пароля;
            \item валидация полей.
        \end{itemize}
    \item Профиль:
        \begin{itemize}
            \item изменение имени;
            \item изменение email;
            \item изменение пароля;
            \item валидация полей;
            \item подписка;
            \item удаление аккаунта.
        \end{itemize}
\end{enumerate}

Выполненные пункты отмечаются следующими статусами: <<пройден успешно>> (\textit{passed}), <<провален>> (\textit{failed}), <<заблокирован>> (\textit{blocked}), <<пропущен>> (\textit{skipped}), <<не был запущен>> (\textit{not run}) \cite{dot-com}.

Для чего используются чек-листы: 
\begin{enumerate}
    \item улучшить представление о системе в целом, видеть статус ее готовности;
    \item понимать объем проделанной и предстоящей работы по тестированию;
    \item не повторяться в проверках и не упустить ничего важного в процессе тестирования.
\end{enumerate}


\subsection{Баги}


\textbf{Баг} (\textit{bug}) -- это документ, описывающий ситуацию или последовательность действий приведшую к некорректной работе объекта тестирования, с указанием причин и ожидаемого результата. \cite{google-testing}

\textbf{Баг репорт} (\textit{but report}) -- это технический документ, причём язык описания проблемы должен быть техническим. Должна использоваться правильная терминология при использовании названий элементов пользовательского интерфейса (например: \textit{editbox, listbox, combobox, link, text area, button, menu, popup menu, title bar, system tray} и т.д.), действий пользователя (например: \textit{click link, press the button, select menu item} и т.д.) и полученных результатах (например: \textit{window is opened, error message is displayed, system crashed} и т.д.). \cite{google-testing}

Структура баг репорта \cite{google-testing}:
\begin{enumerate}[leftmargin=1cm]
    \item \textbf{Шапка}
        \begin{itemize}
            \item Короткое описание (\textit{summary}) -- короткое описание проблемы, явно указывающее на причину и тип ошибочной ситуации.
            \item Проект (\textit{project}) -- название тестируемого проекта.
            \item Компонент приложения (\textit{component}) -- название части или функции тестируемого продукта.
            \item Номер версии (\textit{version}) -- версия на которой была найдена ошибка.
            \item Серьезность (\textit{severity}) -- наиболее распространена пятиуровневая система градации серьезности дефекта, которая описана в таблице \ref{table:severity}.
            \item Приоритет (\textit{priority}) -- приоритет бага, то есть важность его устранения, градация которой описана в таблице \ref{table:priority}.
            \item Статус (\textit{status}) -- cтатус бага. Зависит от используемой процедуры и жизненного цикла бага (\textit{bug workflow and life cycle}).
            \item Автор (\textit{author}) -- cоздатель баг-репорта.
            \item Назначен на (\textit{assigned to}) -- имя сотрудника, назначенного на решение проблемы.
        \end{itemize}
    \item \textbf{Окружение}
        \begin{itemize}
            \item ОС (\textit{OS}) -- операционная система, на которой был обнаружен баг.
            \item Пакет обновления (\textit{service pack}) -- установленный набор обновлений, исправлений или улучшений программного обеспечения, на котором был обнаружен баг.
        \end{itemize}
    \item \textbf{Описание}
        \begin{itemize}
            \item Шаги воспроизведения (\textit{steps to reproduce}) -- шаги, по которым можно легко воспроизвести ситуацию, приведшую к ошибке.
            \item Фактический результат (\textit{result}) -- результат, полученный после прохождения шагов к воспроизведению.
            \item Ожидаемый результат (\textit{expected result}) -- ожидаемый правильный результат.
        \end{itemize}
    \item \textbf{Дополнения}
        \begin{itemize}
            \item Прикрепленный файл (\text{attachment}) -- файл с логами, скриншот или любой другой документ, который может помочь прояснить причину ошибки или указать на способ решения проблемы.
        \end{itemize}
\end{enumerate}

\begin{table}[H]
    \caption{Градация серьёзности дефекта}
    \label{table:severity}
    \setlength{\parskip}{1.0ex}
    \renewcommand{\arraystretch}{1.5}
    \renewcommand{\tabularxcolumn}[1]{m{#1}}
    \begin{tabularx}{\textwidth}{|p{0.05 \textwidth}|X|}
        \hline
        \textbf{S1} & Блокирующий (\textit{blocker})  \\ \hline
        \textbf{S2} & Критический (\textit{critical}) \\ \hline
        \textbf{S3} & Значительный (\textit{major})   \\ \hline
        \textbf{S4} & Незначительный (\textit{minor}) \\ \hline
        \textbf{S5} & Тривиальный (\textit{trivial})  \\ \hline
    \end{tabularx}
\end{table}

\begin{table}[H]
    \caption{Градация приоритета бага}
    \label{table:priority}
    \setlength{\parskip}{1.0ex}
    \renewcommand{\arraystretch}{1.5}
    \renewcommand{\tabularxcolumn}[1]{m{#1}}
    \begin{tabularx}{\textwidth}{|p{0.05 \textwidth}|X|}
        \hline
        \textbf{P1} & Высокий (\textit{high})  \\ \hline
        \textbf{P2} & Средний (\textit{medium}) \\ \hline
        \textbf{P3} & Низкий (\textit{low})   \\ \hline
    \end{tabularx}
\end{table}

Серьезность (\textit{severity}) -- это атрибут, характеризующий влияние дефекта на работоспособность приложения. \cite{google-testing}

Приоритет (\textit{priority}) -- это атрибут, указывающий на очередность выполнения задачи или устранения дефекта. Можно сказать, что это инструмент менеджера по планированию работ. Чем выше приоритет, тем быстрее нужно исправить дефект. \cite{google-testing}

\newpage