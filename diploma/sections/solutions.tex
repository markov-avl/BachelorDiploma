\section{Обзор существующих решений}\label{sec:existing-solutions}

Системы автоматического тестирования делятся на два типа:
\begin{itemize}
    \item системы автоматическо создания тестов;
    \item системы автоматического запуска тестов.
\end{itemize}

Примеры систем автоматического создания тестов:
\begin{itemize}
    \item IntelliTest.
\end{itemize}

Примеры систем автоматического запуска тестов:
\begin{itemize}
    \item PyTest;
    \item Unittest;
    \item GoogleTest.
\end{itemize}

\begin{enumerate}
    \item \textbf{IntelliTest} -- позволяет находить ошибки на раннем этапе и уменьшает затраты на обслуживание тестирования. Благодаря автоматизированному и прозрачному подходу к тестированию инструмент IntelliTest позволяет сформировать набор кандидатов на тесты для кода .NET. Создание набора тестов можно дополнительно настроить с помощью задаваемых вами свойств правильности. IntelliTest даже будет автоматически модифицировать набор тестов по мере модификации тестируемого кода.

    Характеристические тесты. IntelliTest позволяет определить поведение кода в рамках набора традиционных модульных тестов. Подобный набор тестов можно использовать в качестве набора регрессии, чтобы сформировать основу для преодоления сложностей, связанных с рефакторингом незнакомого кода или кода прежних версий.

    Инструмент IntelliTest использует открытый подход к анализу кода и поиску решений для ограничений, чтобы автоматически создавать точные входные значения для тестов, при этом вмешательство пользователя обычно не требуется. Для сложных типов объектов он автоматически создает фабрики. Вы можете управлять созданием входных данных для теста, расширяя и настраивая фабрики в соответствии с вашими требованиями. Свойства правильности, заданные в коде в качестве утверждений, также будут автоматически использоваться для дальнейшего управления созданием входных данных.

    Интеграция со средой IDE. Инструмент IntelliTest полностью интегрирован в среду IDE Visual Studio. Все данные, собранные во время создания набора тестов (например, автоматически созданные входные данные, выходные данные кода, созданные тестовые случаи и состояние их выполнения), отображаются в интегрированной среде разработки Visual Studio. Вы можете легко переключаться между правкой кода и повторным выполнением IntelliTest, не выходя из среды IDE Visual Studio. Тесты можно сохранить в качестве решения -- проекта модульных тестов, после чего их автоматически определит обозреватель тестов Visual Studio.
    
    \item \textbf{PyTest} -- это фреймворк, упрощающий создание простых и масштабируемых тестов. Тесты выразительны и удобочитаемы -- шаблонный код не требуется.

    \item \textbf{Unittest} -- модуль в Python, который поддерживает автоматизацию тестов, использование общего кода для настройки и завершения тестов, объединение тестов в группы, а также позволяет отделять тесты от фреймворка для вывода информации.

    Для автоматизации тестов unittest поддерживает некоторые важные концепции:

    Испытательный стенд (\textit{test fixture}) - выполняется подготовка, необходимая для выполнения тестов и все необходимые действия для очистки после выполнения тестов. Это может включать, например, создание временных баз данных или запуск серверного процесса.

    Тестовый случай (\textit{test case}) - минимальный блок тестирования. Он проверяет ответы для разных наборов данных. Модуль unittest предоставляет базовый класс TestCase, который можно использовать для создания новых тестовых случаев.

    Набор тестов (\textit{test suite}) - несколько тестовых случаев, наборов тестов или и того и другого. Он используется для объединения тестов, которые должны быть выполнены вместе.

    Исполнитель тестов (\textit{test runner}) - компонент, который управляет выполнением тестов и предоставляет пользователю результат. Исполнитель может использовать графический или текстовый интерфейс или возвращать специальное значение, которое сообщает о результатах выполнения тестов.

    Модуль unittest предоставляет богатый набор инструментов для написания и запуска тестов. Однако достаточно лишь некоторых из них, чтобы удовлетворить потребности большинства пользователей.
    
    \item \textbf{GoogleTest} -- библиотека для модульного тестирования на языке С++.

    Google Test построена на методологии тестирования xUnit, то есть когда отдельные части программы (классы, функции, модули) проверяются отдельно друг от друга, в изоляции. Библиотека сама по себе разработана с активным применением тестирования, когда при добавлении каких-либо частей в официальную версию, кроме кода самих изменений необходимо написать набор тестов, подтверждающих их корректность.

\end{enumerate}

% украл у Михеева

\begin{xltabular}[h]{\textwidth}{|p{0.15 \textwidth}|C|C|C|}
    \caption{Сравнение существующих решений\label{tab:solutions-comparison}} \\
    \hline
                & \textbf{Поддержка различных ЯП} & \textbf{Автоматическое создание тестов} & \textbf{Автоматический запуск тестов} \\
    \hline \endhead
    InteliiTest & -                      & +                              & +                            \\ \hline
    GoogleTest  & -                      & -                              & +                            \\ \hline
    UnitTest    & -                      & -                              & +                            \\ \hline
    PyTest      & -                      & -                              & +                            \\ \hline
\end{xltabular}

\newpage