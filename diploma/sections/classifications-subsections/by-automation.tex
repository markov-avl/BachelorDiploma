\subsection{Классификация по степени автоматизации}\label{subsec:by-automation}

\textbf{Ручное тестирование} (\textit{manual testing}) -- тестирование, в котором тесты выполняются человеком вручную без использования средств автоматизации. \cite{aprozorova}

\textbf{Автоматизированное тестирование} (\textit{automated testing}) -- набор техник, подходов и инструментальных средств, позволяющий исключить человека из выполнения некоторых задач в процессе тестирования. \cite{aprozorova} Тест-кейсы частично или полностью выполняет специальное инструментальное средство, однако разработка тест-кейсов, подготовка данных, оценка результатов выполнения, написания отчётов об обнаруженных дефектах -- всё это и многое другое по-прежнему делает вручную.

\textbf{Полуавтоматизированное тестирование} (\textit{semi-automated testing}) -- ручное тестирование с частичным использованием средств автоматизации - программного обеспечения (например, средств захвата/воспроизведения) для контроля выполнения тестов, сравнения полученных результатов с эталонными, установки предусловий тестов и других функций контроля тестирования и организации отчетов. \cite{aprozorova}